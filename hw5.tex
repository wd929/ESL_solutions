\documentclass[11pt, oneside]{article}   	% use "amsart" instead of "article" for AMSLaTeX format
\usepackage{geometry}                		% See geometry.pdf to learn the layout options. There are lots.
\geometry{letterpaper}                   		% ... or a4paper or a5paper or ... 
%\geometry{landscape}                		% Activate for rotated page geometry
%\usepackage[parfill]{parskip}    		% Activate to begin paragraphs with an empty line rather than an indent
\usepackage{graphicx}				% Use pdf, png, jpg, or eps§ with pdflatex; use eps in DVI mode
								% TeX will automatically convert eps --> pdf in pdflatex		
\usepackage{amssymb}

%SetFonts

%SetFonts


\begin{document}
\section{6.2}
%\subsection{}
By the definition of vector-valued function, $b(x)^T = (1, x)$ and $\mathbf{B} = [\mathbf{1}, \mathbf{x}]$, so we can get
\begin{eqnarray}
b(x_0)^T &=& b(x_0)^T (\mathbf{B}^T \mathbf{W}(x_0) \mathbf{B})^{-1} \mathbf{B}^T \mathbf{W}(x_0) \mathbf{B}\\
(1, x_0) &=& b(x_0)^T (\mathbf{B}^T \mathbf{W}(x_0) \mathbf{B})^{-1} \mathbf{B}^T \mathbf{W}(x_0) [\mathbf{1}, \mathbf{x}]
\end{eqnarray}
That means
\begin{eqnarray}
1 &=& b(x_0)^T (\mathbf{B}^T \mathbf{W}(x_0) \mathbf{B})^{-1} \mathbf{B}^T \mathbf{W}(x_0) \mathbf{1} \nonumber \\
&=& \sum_{i = 1}^N l_i(x_0) \label{eq1} 
\end{eqnarray}
and
\begin{eqnarray}
x_0 &=& b(x_0)^T (\mathbf{B}^T \mathbf{W}(x_0) \mathbf{B})^{-1} \mathbf{B}^T \mathbf{W}(x_0) \mathbf{x} \nonumber \\
&=& \sum_{i = 1}^N l_i(x_0)x_i  \label{eq2} 
\end{eqnarray}
\begin{eqnarray}
\sum_{i = 1}^N (x_i - x_0) l_i(x_0) &=& \sum_{i = 1}^N l_i(x_0)x_i - x_o \sum_{i = 1}^N l_i(x_0) \nonumber \\
&=& x_0 - x_0 \cdot 1\\
&=& 0
\end{eqnarray}
From equation \ref{eq1}, we can get
\begin{eqnarray}
b(x_0) = \sum_{i = 1}^N (x_i - x_0)^0 l_i(x_0) = \sum_{i = 1}^N l_i(x_0) = 1
\end{eqnarray}
When $j \in \{2,...,k\}$, we can get vector-valued function $b(x)^T = (1,x,x^2,...,x^k)$ and $ \mathbf{B} = [ \mathbf{1}, \mathbf{x}, \mathbf{x^2},..., \mathbf{x^k}]$. 
From equation \ref{eq2}, we can get
\begin{eqnarray}
x_0^j = \sum_{i = 1}^N l_i(x_0) x_i^j
\end{eqnarray}
Expanding $(x_i - x_0)^j$, each term can be written as $(-1)^b x_i^a x_0^b$, where $a + b = j$. And, it is obvious that the number of positive terms is equal to the negative one, i.e. $\sum_{n=1}^{2^j} (-1)^{b_n} = 0$. So each term of $b_j(x_0)$ can be written as
\begin{eqnarray}
\sum_{i = 1}^N (-1)^b x_i^a x_0^b l_i(x_0) &=& (-1)^b x_0^b \sum_{i = 1}^N l_i(x_0) x_i^a\\
&=& (-1)^b x_0^b x_0^a\\
&=& (-1)^b x_0^j\\
b_j(x_0) &=& \sum_{i = 1}^N (x_i - x_0)^j l_i(x_0)\\
&=& \sum_{n = 1}^{2^j} \sum_{i = 1}^N (-1)^{b_n} x_i^{j - b_n} x_0^{b_n} l_i(x_0) \\
&=& \sum_{n = 1}^{2^j} [ (-1)^{b_n} x_0^j]\\
&=& 0
\end{eqnarray}
So, we have the bias
\begin{eqnarray}
\mathbb{E}[\hat{f}(x_0) - f(x_0)] &=& \sum_{i = 1}^N l_i(x_0)f(x_i) - f(x_0)\\
&=& [f(x_0) \sum_{i = 1}^N l_i(x_0) - f(x_0)] + f^{'}(x_0) \sum_{i = 1}^N (x_i - x_0) l_i(x_0) \nonumber \\
&& + k_2 f^{''}(x_0) \sum_{i = 1}^N (x_i - x_0)^2 l_i(x_0) + ... \nonumber \\
&& + k_{j+1} f^{(j+1)}(x_0) \sum_{i = 1}^N (x_i - x_0)^{(j+1)} l_i(x_0) + ... \\
&=& k_{j+1} f^{(j+1)}(x_0) \sum_{i = 1}^N (x_i - x_0)^{(j+1)} l_i(x_0) + ...
\end{eqnarray}
where $k_n$ is the coefficient of series expansion terms. Now, we can get that the bias depends only on $j+1$th degree and higher order terms in the expansion of $f$.

\section{6.3}
$\mathbf{B}$ is a $N * 2$ regression matrix, which is not invertible, while $\mathbf{B}\mathbf{B}^T$ is invertible. Because,
\begin{eqnarray}
\hat{f}(x_j) &=& b(x_j)^T (\mathbf{B}^T \mathbf{W}(x_j)\mathbf{B})^{-1} \mathbf{B}^T \mathbf{W}(x_j)\mathbf{y}\\
&=&  \sum_{i = 1}^N l_i(x_j)y_i\\
&=& {l}(x_j)^T\mathbf{y}
\end{eqnarray}
We can get
\begin{eqnarray}
{l}(x_j)^T &=& b(x_j)^T (\mathbf{B}^T \mathbf{W}(x_j)\mathbf{B})^{-1} \mathbf{B}^T \mathbf{W}(x_j)\\
{l}(x_j) &=& \mathbf{W}(x_j) \mathbf{B} (\mathbf{B}^T \mathbf{W}(x_j)\mathbf{B})^{-1} b(x_j)
\end{eqnarray}
So, 
\begin{eqnarray}
||l(x_j)||^2 &=& {l}(x_j)^T {l}(x_j)\\
&=& [b(x_j)^T (\mathbf{B}^T \mathbf{W}(x_j)\mathbf{B})^{-1} \mathbf{B}^T \mathbf{W}(x_j)][\mathbf{W}(x_j) \mathbf{B} (\mathbf{B}^T \mathbf{W}(x_j)\mathbf{B})^{-1} b(x_j)]\\
&=& [b(x_j)^T (\mathbf{B}^T \mathbf{W}(x_j)\mathbf{B})^{-1} \mathbf{B}^T \mathbf{W}(x_j)][(\mathbf{B}\mathbf{B}^T) (\mathbf{B}\mathbf{B}^T)^{-1} (\mathbf{B}\mathbf{B}^T)^{-1} (\mathbf{B}\mathbf{B}^T)]\\
&& [\mathbf{W}(x_j) \mathbf{B} (\mathbf{B}^T \mathbf{W}(x_j)\mathbf{B})^{-1} b(x_j)] \nonumber \\
&=& b(x_j)^T \mathbf{B}^T (\mathbf{B}\mathbf{B}^T)^{-1} (\mathbf{B}\mathbf{B}^T)^{-1} \mathbf{B} b(x_j)
\end{eqnarray}
So,
\begin{eqnarray}
||l(x)||^2 &=& \sum_{j = 1}^{d+1} ||l(x_j)||^2\\
&=& \sum_{j = 1}^{d+1}  b(x_j)^T \mathbf{B}^T (\mathbf{B}\mathbf{B}^T)^{-1} (\mathbf{B}\mathbf{B}^T)^{-1} \mathbf{B} b(x_j)\\
&=& \mathrm{trace} (\mathbf{B}\mathbf{B}^T(\mathbf{B}\mathbf{B}^T)^{-1}(\mathbf{B}\mathbf{B}^T)^{-1}(\mathbf{B}\mathbf{B}^T))\\
&=& \mathrm{trace} (\mathbf{I}_{d+1})\\
&=& d+1
\end{eqnarray}
Hence $||l(x+1)|| = \sqrt{d+1}$ which increases with the degree of the local polynomial. 
\end{document} \sum_{i = 1}^N 