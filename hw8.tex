\documentclass[11pt, oneside]{article}   	% use "amsart" instead of "article" for AMSLaTeX format
\usepackage{geometry}                		% See geometry.pdf to learn the layout options. There are lots.
\geometry{a4paper}                   		% ... or a4paper or a5paper or ... 
%\geometry{landscape}                		% Activate for rotated page geometry
%\usepackage[parfill]{parskip}    		% Activate to begin paragraphs with an empty line rather than an indent

\usepackage{indentfirst}
\usepackage[namelimits]{amsmath} 
\usepackage{amssymb}             
\usepackage{amsfonts}            
\usepackage{mathrsfs}

\usepackage{graphicx}

%\usepackage{graphicx}				% Use pdf, png, jpg, or eps§ with pdflatex; use eps in DVI mode
								% TeX will automatically convert eps --> pdf in pdflatex		
\usepackage{amssymb}

%SetFonts

%SetFonts


\begin{document}
We define Plus-plane as {$\mathbf{x} : \mathbf{w} . \mathbf{x} + \mathbf{b} = 1 $} and Minius-plane as  {$\mathbf{x} : \mathbf{w} . \mathbf{x} + \mathbf{b} = -1 $}, shown in the following image.\\
\centerline{\includegraphics[height=0.3\textwidth]{./pic.png} }
 
let $\mathbf{u}$ and $\mathbf{v}$ be two vectors in the Plus-plane, we can get 
\begin{eqnarray}
\mathbf{w} . \mathbf{u} + \mathbf{b} &=& 1 \label{left}\\
\mathbf{w} . \mathbf{v} + \mathbf{b} &=& 1 \label{right}
\end{eqnarray}  
and let eq \ref{left} minus eq \ref{right}, i.e.
\begin{eqnarray}
( \mathbf{w} . \mathbf{u} + \mathbf{b} ) - ( \mathbf{w} . \mathbf{v} + \mathbf{b} ) &=&  0 \\
\mathbf{w} . ( \mathbf{u} - \mathbf{v} ) &=&  0 
\end{eqnarray}  
which means $\mathbf{w}$ is the perpendicular of the Plus-plane. Similarly, the vector $\mathbf{w}$ is also the perpendicular of the Minus-plane. 

If $\mathbf{x^{+}}$ is any point in the Plus-plane and $\mathbf{x^{-}}$ is the closest point to $\mathbf{x^{+}}$ in the Minus-plane, we can also get 
\begin{eqnarray}
\mathbf{w} . \mathbf{x^{+}} + \mathbf{b} &=& 1 \\
\mathbf{w} . \mathbf{x^{-}} + \mathbf{b} &=& -1 
\end{eqnarray}
and $\mathbf{x^{+}} - \mathbf{x^{-}} $ is in the direction of $\mathbf{w}$, which means $\mathbf{x^{+}} - \mathbf{x^{}} = \lambda \mathbf{w}$.
So we can get
\begin{eqnarray}
\mathbf{w} . (\mathbf{x^{-}} + \lambda \mathbf{w}) + \mathbf{b} &=& 1 \\
\mathbf{w} . \mathbf{x^{-}} + \lambda \mathbf{w} . \mathbf{w} + \mathbf{b} &=& 1 \\
-1 + \lambda \mathbf{w} . \mathbf{w} &=& 1 \\
\lambda &=& \frac{2}{\mathbf{w} . \mathbf{w}}
\end{eqnarray}
Define Margin width as $M = |\mathbf{x^{+}} - \mathbf{x^{-}}|$. And\\
\begin{equation}
M = | \mathbf{x^{+}} - \mathbf{x^{-}}| = |\lambda \mathbf{w}| = \lambda |\mathbf{w}| = \lambda \sqrt{\mathbf{w}.\mathbf{w}} = \frac{2}{\sqrt{\mathbf{w} . \mathbf{w}}}
\end{equation}
Let $y_i$ be the label of $x_i$ in our data set. The Maximum Margin problem of the decision boundary can be found by solving the following constrained optimization problem.
\begin{eqnarray}
\mathrm{minimum}\,\frac{1}{2} || \mathbf{w} ||^{2}\\
\mathrm{subject\,to}\ y_i (\mathbf{w} . \mathbf{x}_i + b )&\ge&1
\end{eqnarray}
\end{document} 